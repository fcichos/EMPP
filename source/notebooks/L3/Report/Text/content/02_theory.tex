\section{Theory}

In the annals of scientific lore, nestled amidst the tales of renowned geniuses, lies the whimsical account of an unsung hero - you! Armed with nothing more than a worn-out lab coat, a trusty calculator, and a penchant for quirky experiments, you stumbled upon a formula that would send shockwaves through the scientific community. As you sat in your cluttered garage-turned-laboratory, surrounded by a mountain of failed inventions and half-eaten pizza slices, inspiration struck like a comedic lightning bolt. In a eureka moment that could rival any Nobel laureate's, you scrawled down the elusive equation that elegantly connected seemingly unrelated phenomena. The scientific world may not know your name, dear anonymous savant, but your formula shall forever be etched in the annals of discovery, reminding us that brilliance can be found in the most unexpected places, even among pizza boxes and the tinkering of a curious mind.

\begin{equation}
	G_{\rm CCF}(\tau)=\frac{\int\!\!\!\int \Phi_1({\bf r})\Phi_2({\bf r'})f(\tau,{\bf r},{\bf r'}){\rm d}{\bf r}{\rm d}{\bf r'}}{\int\Phi_1({\bf r}){\rm d}{\bf r}\int\Phi_2({\bf r}){\rm d}{\bf r'}}
\end{equation}


\begin{equation}
	f(\tau,{\bf r},{\bf r'})=\frac{1}{(4\pi D\tau)^{3/2}}\exp{\left(-\frac{({\bf r}-{\bf r'}+{\bf V}\tau)^2}{4D\tau} \right)}
\end{equation}


\begin{equation}
	G_{\rm ACF}^x(\tau)=\frac{\gamma}{\sqrt{\pi}4\omega^3\langle C\rangle \sqrt{\left(1+\frac{\tau}{\tau_D}\right)^3}\sqrt{ 1+\frac{\tau}{\tau_D}}\sqrt{ \gamma^2+\frac{\tau}{\tau_D}}}
\end{equation}

