% Options for packages loaded elsewhere
\PassOptionsToPackage{unicode}{hyperref}
\PassOptionsToPackage{hyphens}{url}
\PassOptionsToPackage{dvipsnames,svgnames,x11names}{xcolor}
%
\documentclass[
  letterpaper,
  DIV=11,
  numbers=noendperiod]{scrartcl}

\usepackage{amsmath,amssymb}
\usepackage{iftex}
\ifPDFTeX
  \usepackage[T1]{fontenc}
  \usepackage[utf8]{inputenc}
  \usepackage{textcomp} % provide euro and other symbols
\else % if luatex or xetex
  \usepackage{unicode-math}
  \defaultfontfeatures{Scale=MatchLowercase}
  \defaultfontfeatures[\rmfamily]{Ligatures=TeX,Scale=1}
\fi
\usepackage{lmodern}
\ifPDFTeX\else  
    % xetex/luatex font selection
\fi
% Use upquote if available, for straight quotes in verbatim environments
\IfFileExists{upquote.sty}{\usepackage{upquote}}{}
\IfFileExists{microtype.sty}{% use microtype if available
  \usepackage[]{microtype}
  \UseMicrotypeSet[protrusion]{basicmath} % disable protrusion for tt fonts
}{}
\makeatletter
\@ifundefined{KOMAClassName}{% if non-KOMA class
  \IfFileExists{parskip.sty}{%
    \usepackage{parskip}
  }{% else
    \setlength{\parindent}{0pt}
    \setlength{\parskip}{6pt plus 2pt minus 1pt}}
}{% if KOMA class
  \KOMAoptions{parskip=half}}
\makeatother
\usepackage{xcolor}
\setlength{\emergencystretch}{3em} % prevent overfull lines
\setcounter{secnumdepth}{-\maxdimen} % remove section numbering
% Make \paragraph and \subparagraph free-standing
\makeatletter
\ifx\paragraph\undefined\else
  \let\oldparagraph\paragraph
  \renewcommand{\paragraph}{
    \@ifstar
      \xxxParagraphStar
      \xxxParagraphNoStar
  }
  \newcommand{\xxxParagraphStar}[1]{\oldparagraph*{#1}\mbox{}}
  \newcommand{\xxxParagraphNoStar}[1]{\oldparagraph{#1}\mbox{}}
\fi
\ifx\subparagraph\undefined\else
  \let\oldsubparagraph\subparagraph
  \renewcommand{\subparagraph}{
    \@ifstar
      \xxxSubParagraphStar
      \xxxSubParagraphNoStar
  }
  \newcommand{\xxxSubParagraphStar}[1]{\oldsubparagraph*{#1}\mbox{}}
  \newcommand{\xxxSubParagraphNoStar}[1]{\oldsubparagraph{#1}\mbox{}}
\fi
\makeatother

\usepackage{color}
\usepackage{fancyvrb}
\newcommand{\VerbBar}{|}
\newcommand{\VERB}{\Verb[commandchars=\\\{\}]}
\DefineVerbatimEnvironment{Highlighting}{Verbatim}{commandchars=\\\{\}}
% Add ',fontsize=\small' for more characters per line
\usepackage{framed}
\definecolor{shadecolor}{RGB}{241,243,245}
\newenvironment{Shaded}{\begin{snugshade}}{\end{snugshade}}
\newcommand{\AlertTok}[1]{\textcolor[rgb]{0.68,0.00,0.00}{#1}}
\newcommand{\AnnotationTok}[1]{\textcolor[rgb]{0.37,0.37,0.37}{#1}}
\newcommand{\AttributeTok}[1]{\textcolor[rgb]{0.40,0.45,0.13}{#1}}
\newcommand{\BaseNTok}[1]{\textcolor[rgb]{0.68,0.00,0.00}{#1}}
\newcommand{\BuiltInTok}[1]{\textcolor[rgb]{0.00,0.23,0.31}{#1}}
\newcommand{\CharTok}[1]{\textcolor[rgb]{0.13,0.47,0.30}{#1}}
\newcommand{\CommentTok}[1]{\textcolor[rgb]{0.37,0.37,0.37}{#1}}
\newcommand{\CommentVarTok}[1]{\textcolor[rgb]{0.37,0.37,0.37}{\textit{#1}}}
\newcommand{\ConstantTok}[1]{\textcolor[rgb]{0.56,0.35,0.01}{#1}}
\newcommand{\ControlFlowTok}[1]{\textcolor[rgb]{0.00,0.23,0.31}{\textbf{#1}}}
\newcommand{\DataTypeTok}[1]{\textcolor[rgb]{0.68,0.00,0.00}{#1}}
\newcommand{\DecValTok}[1]{\textcolor[rgb]{0.68,0.00,0.00}{#1}}
\newcommand{\DocumentationTok}[1]{\textcolor[rgb]{0.37,0.37,0.37}{\textit{#1}}}
\newcommand{\ErrorTok}[1]{\textcolor[rgb]{0.68,0.00,0.00}{#1}}
\newcommand{\ExtensionTok}[1]{\textcolor[rgb]{0.00,0.23,0.31}{#1}}
\newcommand{\FloatTok}[1]{\textcolor[rgb]{0.68,0.00,0.00}{#1}}
\newcommand{\FunctionTok}[1]{\textcolor[rgb]{0.28,0.35,0.67}{#1}}
\newcommand{\ImportTok}[1]{\textcolor[rgb]{0.00,0.46,0.62}{#1}}
\newcommand{\InformationTok}[1]{\textcolor[rgb]{0.37,0.37,0.37}{#1}}
\newcommand{\KeywordTok}[1]{\textcolor[rgb]{0.00,0.23,0.31}{\textbf{#1}}}
\newcommand{\NormalTok}[1]{\textcolor[rgb]{0.00,0.23,0.31}{#1}}
\newcommand{\OperatorTok}[1]{\textcolor[rgb]{0.37,0.37,0.37}{#1}}
\newcommand{\OtherTok}[1]{\textcolor[rgb]{0.00,0.23,0.31}{#1}}
\newcommand{\PreprocessorTok}[1]{\textcolor[rgb]{0.68,0.00,0.00}{#1}}
\newcommand{\RegionMarkerTok}[1]{\textcolor[rgb]{0.00,0.23,0.31}{#1}}
\newcommand{\SpecialCharTok}[1]{\textcolor[rgb]{0.37,0.37,0.37}{#1}}
\newcommand{\SpecialStringTok}[1]{\textcolor[rgb]{0.13,0.47,0.30}{#1}}
\newcommand{\StringTok}[1]{\textcolor[rgb]{0.13,0.47,0.30}{#1}}
\newcommand{\VariableTok}[1]{\textcolor[rgb]{0.07,0.07,0.07}{#1}}
\newcommand{\VerbatimStringTok}[1]{\textcolor[rgb]{0.13,0.47,0.30}{#1}}
\newcommand{\WarningTok}[1]{\textcolor[rgb]{0.37,0.37,0.37}{\textit{#1}}}

\providecommand{\tightlist}{%
  \setlength{\itemsep}{0pt}\setlength{\parskip}{0pt}}\usepackage{longtable,booktabs,array}
\usepackage{calc} % for calculating minipage widths
% Correct order of tables after \paragraph or \subparagraph
\usepackage{etoolbox}
\makeatletter
\patchcmd\longtable{\par}{\if@noskipsec\mbox{}\fi\par}{}{}
\makeatother
% Allow footnotes in longtable head/foot
\IfFileExists{footnotehyper.sty}{\usepackage{footnotehyper}}{\usepackage{footnote}}
\makesavenoteenv{longtable}
\usepackage{graphicx}
\makeatletter
\def\maxwidth{\ifdim\Gin@nat@width>\linewidth\linewidth\else\Gin@nat@width\fi}
\def\maxheight{\ifdim\Gin@nat@height>\textheight\textheight\else\Gin@nat@height\fi}
\makeatother
% Scale images if necessary, so that they will not overflow the page
% margins by default, and it is still possible to overwrite the defaults
% using explicit options in \includegraphics[width, height, ...]{}
\setkeys{Gin}{width=\maxwidth,height=\maxheight,keepaspectratio}
% Set default figure placement to htbp
\makeatletter
\def\fps@figure{htbp}
\makeatother

\KOMAoption{captions}{tableheading}
\makeatletter
\@ifpackageloaded{tcolorbox}{}{\usepackage[skins,breakable]{tcolorbox}}
\@ifpackageloaded{fontawesome5}{}{\usepackage{fontawesome5}}
\definecolor{quarto-callout-color}{HTML}{909090}
\definecolor{quarto-callout-note-color}{HTML}{0758E5}
\definecolor{quarto-callout-important-color}{HTML}{CC1914}
\definecolor{quarto-callout-warning-color}{HTML}{EB9113}
\definecolor{quarto-callout-tip-color}{HTML}{00A047}
\definecolor{quarto-callout-caution-color}{HTML}{FC5300}
\definecolor{quarto-callout-color-frame}{HTML}{acacac}
\definecolor{quarto-callout-note-color-frame}{HTML}{4582ec}
\definecolor{quarto-callout-important-color-frame}{HTML}{d9534f}
\definecolor{quarto-callout-warning-color-frame}{HTML}{f0ad4e}
\definecolor{quarto-callout-tip-color-frame}{HTML}{02b875}
\definecolor{quarto-callout-caution-color-frame}{HTML}{fd7e14}
\makeatother
\makeatletter
\@ifpackageloaded{caption}{}{\usepackage{caption}}
\AtBeginDocument{%
\ifdefined\contentsname
  \renewcommand*\contentsname{Table of contents}
\else
  \newcommand\contentsname{Table of contents}
\fi
\ifdefined\listfigurename
  \renewcommand*\listfigurename{List of Figures}
\else
  \newcommand\listfigurename{List of Figures}
\fi
\ifdefined\listtablename
  \renewcommand*\listtablename{List of Tables}
\else
  \newcommand\listtablename{List of Tables}
\fi
\ifdefined\figurename
  \renewcommand*\figurename{Figure}
\else
  \newcommand\figurename{Figure}
\fi
\ifdefined\tablename
  \renewcommand*\tablename{Table}
\else
  \newcommand\tablename{Table}
\fi
}
\@ifpackageloaded{float}{}{\usepackage{float}}
\floatstyle{ruled}
\@ifundefined{c@chapter}{\newfloat{codelisting}{h}{lop}}{\newfloat{codelisting}{h}{lop}[chapter]}
\floatname{codelisting}{Listing}
\newcommand*\listoflistings{\listof{codelisting}{List of Listings}}
\makeatother
\makeatletter
\makeatother
\makeatletter
\@ifpackageloaded{caption}{}{\usepackage{caption}}
\@ifpackageloaded{subcaption}{}{\usepackage{subcaption}}
\makeatother

\ifLuaTeX
  \usepackage{selnolig}  % disable illegal ligatures
\fi
\usepackage{bookmark}

\IfFileExists{xurl.sty}{\usepackage{xurl}}{} % add URL line breaks if available
\urlstyle{same} % disable monospaced font for URLs
\hypersetup{
  pdftitle={Brownian Motion},
  colorlinks=true,
  linkcolor={blue},
  filecolor={Maroon},
  citecolor={Blue},
  urlcolor={Blue},
  pdfcreator={LaTeX via pandoc}}


\title{Brownian Motion}
\author{}
\date{}

\begin{document}
\maketitle


We will apply our newly acquired knowledge about classes to simulate
Brownian motion. This task aligns perfectly with the principles of
object-oriented programming, as each Brownian particle (or colloid) can
be represented as an object instantiated from the same class, albeit
with different properties. For instance, some particles might be larger
while others are smaller. We have already touched on some aspects of
this in previous lectures.

\begin{Shaded}
\begin{Highlighting}[]
\NormalTok{\#| autorun: true}
\NormalTok{\#| edit: false}
\NormalTok{\#| echo: false}
\NormalTok{import numpy as np}
\NormalTok{import matplotlib.pyplot as plt}
\NormalTok{import pandas as pd}

\NormalTok{plt.rcParams.update(\{\textquotesingle{}font.size\textquotesingle{}: 12,}
\NormalTok{                     \textquotesingle{}lines.linewidth\textquotesingle{}: 1,}
\NormalTok{                     \textquotesingle{}lines.markersize\textquotesingle{}: 10,}
\NormalTok{                     \textquotesingle{}axes.labelsize\textquotesingle{}: 11,}
\NormalTok{                     \textquotesingle{}xtick.labelsize\textquotesingle{} : 10,}
\NormalTok{                     \textquotesingle{}ytick.labelsize\textquotesingle{} : 10,}
\NormalTok{                     \textquotesingle{}xtick.top\textquotesingle{} : True,}
\NormalTok{                     \textquotesingle{}xtick.direction\textquotesingle{} : \textquotesingle{}in\textquotesingle{},}
\NormalTok{                     \textquotesingle{}ytick.right\textquotesingle{} : True,}
\NormalTok{                     \textquotesingle{}ytick.direction\textquotesingle{} : \textquotesingle{}in\textquotesingle{},\})}
\end{Highlighting}
\end{Shaded}

\subsection{Brownian Motion}\label{brownian-motion}

\subsubsection{What is Brownian Motion?}\label{what-is-brownian-motion}

Imagine a dust particle floating in water. If you look at it under a
microscope, you'll see it moving in a random, zigzag pattern. This is
Brownian motion!

\subsubsection{Why Does This Happen?}\label{why-does-this-happen}

When we observe Brownian motion, we're seeing the effects of countless
molecular collisions. Water isn't just a smooth, continuous fluid - it's
made up of countless tiny molecules that are in constant motion. These
water molecules are continuously colliding with our particle from all
directions. Each individual collision causes the particle to move just a
tiny bit, barely noticeable on its own. However, when millions of these
tiny collisions happen every second from random directions, they create
the distinctive zigzag motion we observe.

\phantomsection\label{fig_brownian}
\begin{center}
\includegraphics[width=0.4\textwidth,height=\textheight]{img/Brownian Motion.gif}
\end{center}

Animation of Brownian motion (c) Wikipedia.

\subsubsection{The Simplified Math Behind
It}\label{the-simplified-math-behind-it}

When our particle moves:

\begin{enumerate}
\def\labelenumi{\arabic{enumi}.}
\tightlist
\item
  Each step is random in direction
\item
  The size of each step depends on:

  \begin{itemize}
  \tightlist
  \item
    Temperature (warmer = more movement)
  \item
    Time between steps
  \item
    A property called the ``diffusion coefficient'' (D)
  \end{itemize}
\end{enumerate}

\subsubsection{How We Can Simulate
This?}\label{how-we-can-simulate-this}

In Python, we can simulate these random steps using random number. These
random numbers can be generated with the numpy library. Numpy provides a
number of different functions that provide random numbers from different
distributions. For Brownian motion, we use a special distribution called
the ``normal distribution''.

\begin{Shaded}
\begin{Highlighting}[]
\NormalTok{step\_size }\OperatorTok{=}\NormalTok{ np.sqrt(}\DecValTok{2} \OperatorTok{*}\NormalTok{ D }\OperatorTok{*}\NormalTok{ time\_step)}
\NormalTok{dx }\OperatorTok{=}\NormalTok{ random\_number }\OperatorTok{*}\NormalTok{ step\_size  }\CommentTok{\# Random step in x direction}
\NormalTok{dy }\OperatorTok{=}\NormalTok{ random\_number }\OperatorTok{*}\NormalTok{ step\_size  }\CommentTok{\# Random step in y direction}

\NormalTok{new\_x }\OperatorTok{=}\NormalTok{ old\_x }\OperatorTok{+}\NormalTok{ dx}
\NormalTok{new\_y }\OperatorTok{=}\NormalTok{ old\_y }\OperatorTok{+}\NormalTok{ dy}
\end{Highlighting}
\end{Shaded}

Where:

\begin{itemize}
\tightlist
\item
  \texttt{D} is how easily the particle moves (diffusion coefficient)
\item
  \texttt{time\_step} is how often we update the position
\item
  \texttt{random\_number} is chosen from a special ``normal
  distribution''
\end{itemize}

\begin{tcolorbox}[enhanced jigsaw, titlerule=0mm, opacityback=0, opacitybacktitle=0.6, bottomtitle=1mm, rightrule=.15mm, breakable, colbacktitle=quarto-callout-tip-color!10!white, coltitle=black, colback=white, arc=.35mm, leftrule=.75mm, toptitle=1mm, title=\textcolor{quarto-callout-tip-color}{\faLightbulb}\hspace{0.5em}{Tip}, colframe=quarto-callout-tip-color-frame, bottomrule=.15mm, toprule=.15mm, left=2mm]

When simulating Brownian motion, we use \texttt{np.random.normal} to
generate random steps following this distribution. The normal
distribution is characterized by two parameters: the mean and the
standard deviation. The mean is the average value, and the standard
deviation is a measure of how spread out the values are. For Brownian
motion, we use a standard deviation that depends on the diffusion
coefficient and the time step. The standard deviation
\(\sigma=\sqrt{2D \Delta t}\) determines the typical step size, which we
can use as a parameter in the normal distribution.

\end{tcolorbox}

\begin{Shaded}
\begin{Highlighting}[]
\NormalTok{\#| autorun: false}

\NormalTok{\# some space to test out some of the random numbers}



\end{Highlighting}
\end{Shaded}

\begin{tcolorbox}[enhanced jigsaw, titlerule=0mm, opacityback=0, opacitybacktitle=0.6, bottomtitle=1mm, rightrule=.15mm, breakable, colbacktitle=quarto-callout-note-color!10!white, coltitle=black, colback=white, arc=.35mm, leftrule=.75mm, toptitle=1mm, title=\textcolor{quarto-callout-note-color}{\faInfo}\hspace{0.5em}{Advanced Mathematical Details}, colframe=quarto-callout-note-color-frame, bottomrule=.15mm, toprule=.15mm, left=2mm]

The Brownian motion of a colloidal particle results from collisions with
surrounding solvent molecules. These collisions lead to a probability
distribution described by:

\[
p(x,\Delta t)=\frac{1}{\sqrt{4\pi D \Delta t}}e^{-\frac{x^2}{4D \Delta t}}
\]

where: - \(D\) is the diffusion coefficient - \(\Delta t\) is the time
step - The variance is \(\sigma^2=2D \Delta t\)

This distribution emerges from the \textbf{central limit theorem}, as
shown by Lindenberg and Lévy, when considering many infinitesimally
small random steps.

The evolution of the probability density function \(p(x,t)\) is governed
by the diffusion equation:

\[
\frac{\partial p}{\partial t}=D\frac{\partial^2 p}{\partial x^2}
\]

This partial differential equation, also known as Fick's second law,
describes how the concentration of particles evolves over time due to
diffusive processes. The Gaussian distribution above is the fundamental
solution (Green's function) of this diffusion equation, representing how
an initially localized distribution spreads out over time.

The connection between the microscopic random motion and the macroscopic
diffusion equation was first established by Einstein in his 1905 paper
on Brownian motion, providing one of the earliest quantitative links
between statistical mechanics and thermodynamics.

\end{tcolorbox}

\subsection{Why Use a Class?}\label{why-use-a-class}

A class is perfect for this physics simulation because each colloidal
particle:

\begin{enumerate}
\def\labelenumi{\arabic{enumi}.}
\tightlist
\item
  Has specific properties

  \begin{itemize}
  \tightlist
  \item
    Size (radius)
  \item
    Current position
  \item
    Movement history
  \item
    Diffusion coefficient
  \end{itemize}
\item
  Follows certain behaviors

  \begin{itemize}
  \tightlist
  \item
    Moves randomly (Brownian motion)
  \item
    Updates its position over time
  \item
    Keeps track of where it's been
  \end{itemize}
\item
  Can exist alongside other particles

  \begin{itemize}
  \tightlist
  \item
    Many particles can move independently
  \item
    Each particle keeps track of its own properties
  \item
    Particles can have different sizes
  \end{itemize}
\item
  Needs to track its state over time

  \begin{itemize}
  \tightlist
  \item
    Remember previous positions
  \item
    Calculate distances moved
  \item
    Maintain its own trajectory
  \end{itemize}
\end{enumerate}

This natural mapping between real particles and code objects makes
classes an ideal choice for our simulation.

\subsection{Class Design}\label{class-design}

Let's design a Python class to simulate colloidal particles undergoing
Brownian motion. This object-oriented approach will help us manage
multiple particles with different properties and behaviors.

\subsubsection{Class-Level Properties}\label{class-level-properties}

The \texttt{Colloid} class will maintain information shared by all
particles:

\begin{enumerate}
\def\labelenumi{\arabic{enumi}.}
\tightlist
\item
  A counter for the total number of particles
\item
  The physical constant \(k_B T/(6\pi\eta) = 2.2×10^{-19}\) (combining
  temperature and fluid properties)
\end{enumerate}

\subsubsection{Class Methods}\label{class-methods}

The class will provide these shared functions:

\begin{enumerate}
\def\labelenumi{\arabic{enumi}.}
\tightlist
\item
  \texttt{how\_many()}: Reports the total number of particles
\item
  \texttt{\_\_str\_\_}: Creates a readable description of a particle's
  properties
\end{enumerate}

\subsubsection{Instance Properties}\label{instance-properties}

Each individual particle object will have:

\begin{enumerate}
\def\labelenumi{\arabic{enumi}.}
\tightlist
\item
  Radius (R)
\item
  Position history (x and y coordinates)
\item
  Unique identifier (index)
\item
  Diffusion coefficient (\(D = k_B T/(6\pi\eta R)\))
\end{enumerate}

\subsubsection{Instance Methods}\label{instance-methods}

Each particle will be able to:

\begin{enumerate}
\def\labelenumi{\arabic{enumi}.}
\tightlist
\item
  \texttt{sim\_trajectory()}: Generate a complete motion path
\item
  \texttt{update(dt)}: Calculate one step of Brownian motion
\item
  \texttt{get\_trajectory()}: Return its movement history
\item
  \texttt{get\_D()}: Provide its diffusion coefficient
\end{enumerate}

\begin{Shaded}
\begin{Highlighting}[]
\NormalTok{\#| autorun: false}
\NormalTok{\# Class definition}
\NormalTok{class Colloid:}

\NormalTok{    \# A class variable, counting the number of Colloids}
\NormalTok{    number = 0}
\NormalTok{    f = 2.2e{-}19 \# this is k\_B T/(6 pi eta) in m\^{}3/s}

\NormalTok{    \# constructor}
\NormalTok{    def \_\_init\_\_(self,R, x0=0, y0=0):}
\NormalTok{        \# add initialisation code here}
\NormalTok{        self.R=R}
\NormalTok{        self.x=[x0]}
\NormalTok{        self.y=[y0]}
\NormalTok{        Colloid.number=Colloid.number+1}
\NormalTok{        self.index=Colloid.number}
\NormalTok{        self.D=Colloid.f/self.R}

\NormalTok{    def get\_D(self):}
\NormalTok{        return(self.D)}

\NormalTok{    def sim\_trajectory(self,N,dt):}
\NormalTok{        for i in range(N):}
\NormalTok{            self.update(dt)}

\NormalTok{    def update(self,dt):}
\NormalTok{        self.x.append(self.x[{-}1]+np.random.normal(0.0, np.sqrt(2*self.D*dt)))}
\NormalTok{        self.y.append(self.y[{-}1]+np.random.normal(0.0, np.sqrt(2*self.D*dt)))}
\NormalTok{        return(self.x[{-}1],self.y[{-}1])}

\NormalTok{    def get\_trajectory(self):}
\NormalTok{        return(pd.DataFrame(\{\textquotesingle{}x\textquotesingle{}:self.x,\textquotesingle{}y\textquotesingle{}:self.y\}))}

\NormalTok{    \# class method accessing a class variable}
\NormalTok{    @classmethod}
\NormalTok{    def how\_many(cls):}
\NormalTok{        return(Colloid.number)}

\NormalTok{    \# insert something that prints the particle position in a formatted way when printing}
\NormalTok{    def \_\_str\_\_(self):}
\NormalTok{        return("I\textquotesingle{}m a particle with radius R=\{0:0.3e\} at x=\{1:0.3e\},y=\{2:0.3e\}.".format(self.R, self.x[{-}1], self.y[{-}1]))}
\end{Highlighting}
\end{Shaded}

\begin{tcolorbox}[enhanced jigsaw, titlerule=0mm, opacityback=0, opacitybacktitle=0.6, bottomtitle=1mm, rightrule=.15mm, breakable, colbacktitle=quarto-callout-note-color!10!white, coltitle=black, colback=white, arc=.35mm, leftrule=.75mm, toptitle=1mm, title=\textcolor{quarto-callout-note-color}{\faInfo}\hspace{0.5em}{Note}, colframe=quarto-callout-note-color-frame, bottomrule=.15mm, toprule=.15mm, left=2mm]

Note that the function \texttt{sim\_trajectory} is actually calling the
function \texttt{update} of the same object to generate the whole
trajectory at once.

\end{tcolorbox}

\subsection{Simulating}\label{simulating}

With the help of this Colloid class, we would like to carry out
simulations of Brownian motion of multiple particles. The simulations
shall

\begin{itemize}
\tightlist
\item
  take n=200 particles
\item
  have N=200 trajectory points each
\item
  start all at 0,0
\item
  particle objects should be stored in a list p\_list
\end{itemize}

\begin{Shaded}
\begin{Highlighting}[]
\NormalTok{\#| autorun: false}
\NormalTok{N=200 \# the number of trajectory points}
\NormalTok{n=200 \# the number of particles}

\NormalTok{p\_list=[]}
\NormalTok{dt=0.05}

\NormalTok{\# creating all objects}
\NormalTok{for i in range(n):}
\NormalTok{    p\_list.append(Colloid(1e{-}6))}


\NormalTok{for (index,p) in enumerate(p\_list):}
\NormalTok{    p.sim\_trajectory(N,dt)}
\end{Highlighting}
\end{Shaded}

\begin{Shaded}
\begin{Highlighting}[]
\NormalTok{\#| autorun: false}
\NormalTok{print(p\_list[42])}
\end{Highlighting}
\end{Shaded}

\subsection{Plotting the trajectories}\label{plotting-the-trajectories}

The next step is to plot all the trajectories.

\begin{Shaded}
\begin{Highlighting}[]
\NormalTok{\#| autorun: false}
\NormalTok{\# we take real world diffusion coefficients so scale up the data to avoid nasty exponentials}
\NormalTok{scale=1e6}

\NormalTok{plt.figure(figsize=(4,4))}

\NormalTok{[plt.plot(np.array(p.x[:])*scale,np.array(p.y[:])*scale,\textquotesingle{}k{-}\textquotesingle{},alpha=0.1,lw=1) for p in p\_list]}
\NormalTok{plt.xlim({-}10,10)}
\NormalTok{plt.ylim({-}10,10)}
\NormalTok{plt.xlabel(\textquotesingle{}x [µm]\textquotesingle{})}
\NormalTok{plt.ylabel(\textquotesingle{}y [µm]\textquotesingle{})}
\NormalTok{plt.tight\_layout()}
\NormalTok{plt.show()}
\end{Highlighting}
\end{Shaded}

\subsection{Characterizing the Brownian
motion}\label{characterizing-the-brownian-motion}

Now that we have a number of trajectories, we can analyze the motion of
our Brownian particles.

\subsubsection{Calculate the particle
speed}\label{calculate-the-particle-speed}

One way is to calculate its speed by measuring how far it traveled
within a certain time \(n\, dt\), where \(dt\) is the timestep of out
simulation. We can do that as

\begin{equation}
v(n dt) = \frac{<\sqrt{(x_{i+n}-x_{i})^2+(y_{i+n}-y_{i})^2}>}{n\,dt}
\end{equation}

The angular brackets on the top take care of the fact that we can
measure the distance traveled within a certain time \(n\, dt\) several
times along a trajectory.

These values can be used to calculate a mean speed. Note that there is
not an equal amount of data pairs for all separations available. For
\(n=1\) there are 5 distances available. For \(n=5\), however, only 1.
This changes the statistical accuracy of the mean.

\begin{Shaded}
\begin{Highlighting}[]
\NormalTok{\#| autorun: false}
\NormalTok{time=np.array(range(1,N))*dt}

\NormalTok{plt.figure(figsize=(4,4))}
\NormalTok{for j in range(100):}
\NormalTok{    t=p\_list[j].get\_trajectory()}
\NormalTok{    md=[np.mean(np.sqrt(t.x.diff(i)**2+t.y.diff(i)**2)) for i in range(1,N)]}
\NormalTok{    md=md/time}
\NormalTok{    plt.plot(time,md,alpha=0.4)}

\NormalTok{plt.ylabel(\textquotesingle{}speed [m/s]\textquotesingle{})}
\NormalTok{plt.xlabel(\textquotesingle{}time [s]\textquotesingle{})}
\NormalTok{plt.tight\_layout()}
\NormalTok{plt.show()}
\end{Highlighting}
\end{Shaded}

The result of this analysis shows, that each particle has an apparent
speed which seems to increase with decreasing time of observation or
which decreases with increasing time. This would mean that there is some
friction at work, which slows down the particle in time, but this is
apparently not true. Also an infinite speed at zero time appears to be
unphysical. The correct answer is just that the speed is no good measure
to characterize the motion of a Brownian particle.

\subsubsection{Calculate the particle mean squared
displacement}\label{calculate-the-particle-mean-squared-displacement}

A better way to characterize the motion of a Brownian particle is the
mean squared displacement, as we have already mentioned it in previous
lectures. We may compare our simulation now to the theoretical
prediction, which is

\begin{equation}
\langle \Delta r^{2}(t)\rangle=2 d D t
\end{equation}

where \(d\) is the dimension of the random walk, which is \(d=2\) in our
case.

\begin{Shaded}
\begin{Highlighting}[]
\NormalTok{\#| autorun: false}
\NormalTok{time=np.array(range(1,N))*dt}

\NormalTok{plt.figure(figsize=(4,4))}
\NormalTok{for j in range(100):}
\NormalTok{    t=p\_list[j].get\_trajectory()}
\NormalTok{    msd=[np.mean(t.x.diff(i).dropna()**2+t.y.diff(i).dropna()**2) for i in range(1,N)]}
\NormalTok{    plt.plot(time,msd,alpha=0.4)}


\NormalTok{plt.plot(time, 4*p\_list[0].D*time,\textquotesingle{}k{-}{-}\textquotesingle{},lw=2,label=\textquotesingle{}theory\textquotesingle{})}
\NormalTok{plt.legend()}
\NormalTok{plt.xlabel(\textquotesingle{}time [s]\textquotesingle{})}
\NormalTok{plt.ylabel(\textquotesingle{}msd $[m\^{}2/s]$\textquotesingle{})}
\NormalTok{plt.tight\_layout()}
\NormalTok{plt.show()}
\end{Highlighting}
\end{Shaded}

The results show that the mean squared displacement of the individual
particles follows \emph{on average} the theoretical predictions of a
linear growth in time. That means, we are able to read the diffusion
coefficient from the slope of the MSD of the individual particles if
recorded in a simulation or an experiment.

Yet, each individual MSD is deviating strongly from the theoretical
prediction especially at large times. This is due to the fact mentioned
earlier that our simulation (or experimental) data only has a limited
number of data points, while the theoretical prediction is made for the
limit of infinite data points.

\begin{tcolorbox}[enhanced jigsaw, titlerule=0mm, opacityback=0, opacitybacktitle=0.6, bottomtitle=1mm, rightrule=.15mm, breakable, colbacktitle=quarto-callout-warning-color!10!white, coltitle=black, colback=white, arc=.35mm, leftrule=.75mm, toptitle=1mm, title=\textcolor{quarto-callout-warning-color}{\faExclamationTriangle}\hspace{0.5em}{Analysis of MSD data}, colframe=quarto-callout-warning-color-frame, bottomrule=.15mm, toprule=.15mm, left=2mm]

Single particle tracking, either in the experiment or in numerical
simulations can therefore only deliver an estimate of the diffusion
coefficient and care should be taken when using the whole MSD to obtain
the diffusion coefficient. One typically uses only a short fraction of
the whole MSD data at short times.

\end{tcolorbox}




\end{document}
